% !Mode:: "TeX:UTF-8"
\chapter{介绍}
这个模板的部分代码来自管彬等人在书写自己毕业论文时候使用的\LaTeX{}代码,这部分主要有声明和页眉页脚。其他更多的涉及到格式细节的内容,则有很多来自于Ch'en Meng和我早先使用\LaTeX{}书写的代码。这些工作很零碎,并且没有完整的用户文档,不利于维护和发布,因此在2013年5月,Ch'en Meng在GitHub上建立了项目\footnote{\url{https://github.com/ChenMeng0518/sduthesis/}}。

最初Knuth 设计开发\TeX{} 的时候没有考虑到支持多国语言,特别是多字节的中日韩
语言。这使得\TeX{} 以至后来的\LaTeX{} 对中文的支持一直不是很好。即使在CJK 解决了中
文字符处理的问题以后,中文用户使用\LaTeX{} 仍然要面对许多困难。最常见的就是中文化
的标题。由于中文习惯和西方语言的不同,使得很难直接使用原有的标题结构来表示中文
标题。因此需要对标准\LaTeX{} 宏包做较大的修改。此外,还有诸如中文字号的对应关系等
等。\texttt{ctex.org}论坛\footnote{\url{http://www.ctex.org/}}网站制作并负责维护的\texttt{ctex}宏包已经较好地解决了这些问题,因此从本模板的2.0.0版本开始,放弃了自己编写的\emph{重复的车轮},在\texttt{ctex}宏包的基础上,重构了全部代码。

这份文档可以使用\XeLaTeX{}编译\texttt{SDUthesistemplate.tex}得到。
\chapter{中文支持}
模板会根据使用的引擎\footnote{\XeTeX{}或者pdf\TeX{}.}自动调用CJK或者xeCJK宏包,因此你无需再显式地调用。此外在使用CJK方式支持中文的时候,模板会在\verb|\begin{document}|和\verb|\end{document}|之间自动插入CJK环境,因此你也不需要重复加入,直接在其中输入中文就能得到正确输出。

由于模板内部编码是UTF8, 所以不论使用\XeLaTeX{}还是使用pdf\LaTeX{}, 都\textit{必须用UTF8编码}保存你的文档,不然会出错。
\chapter{选项}
\section{中文字库}
\label{sec:fontset}
下面的选项用于选择可用的中文字库。设置这些选项是考虑到不同的操作系统平台提
供的中文字库是不同的。

\begin{center}
\begin{tabular}{p{.2\textwidth}p{.7\textwidth}}
\toprule
\textbf{winfonts}& 使用 Windows 的字体设置,默认为六种中易字体:宋体、仿宋、黑体、楷
体、隶书、幼圆(在使用 \XeTeX 时只有前四种)。\textit{这是默认设置}。\\
\textbf{adobefonts}& 在 xeCJK 模式中使用 Adobe 的四套字体:宋体、仿宋、黑体、楷体\footnote{可以在这里下载:\url{http://ishare.iask.sina.com.cn/f/15105086.html}.}。
在 CJK 模式(即不使用 \XeTeX 时)下,该选项将使用 \textbf{winfonts} 选项的设置。\\
\textbf{nofonts}& 没有中文字库,此时没有中文字体命令可用。如果期望使用自己设置的字
体,可以选中这个选项。\\
\bottomrule
\end{tabular}
\end{center}
\section{排版风格}
\begin{center}
\begin{tabular}{p{.2\textwidth}p{.7\textwidth}}
\toprule
\textbf{chsstyle}& 使用中文的标题样式。这个命令会修改章节的标题样式,以及图标目录等的标题。\textit{这是默认设置}。\\
\textbf{nochsstyle}& 保留使用英文的标题样式。\\
\textbf{punct}& 对中文标点的位置(宽度)进行调整。\textit{这是默认设置}。\\
\textbf{nopunct}& 不对中文标点的位置进行调整(每个标点占有相同的宽度)。\\
\bottomrule
\end{tabular}
\end{center}
\section{打印}
\begin{center}
\begin{tabular}{p{.2\textwidth}p{.7\textwidth}}
\toprule
\textbf{noprint}& 保留链接等的色彩,页边距左右相等,\verb|chapter|不强制从偶数页开始。\textit{这是默认设置}。\\
\textbf{print}& 单面打印模式。不保留链接等的色彩,页边距左右相等,\verb|chapter|不强制从偶数页开始。\\
\textbf{dprint}& 双面打印模式。不保留链接等的色彩,页边距左右不相等,\verb|chapter|强制从偶数页开始。\\
\bottomrule
\end{tabular}
\end{center}
\section{默认选项}
总结一下,\verb|sduthesis|的默认选项有:\begin{inparaitem}
\item winfonts\item chsstyle\item punct\item noprint
\end{inparaitem}.
\chapter{常用命令}
\verb|sduthesis|提供了一系列命令,用于修改字体、字号、数字等的呈现形式。
\section{字体}
中文字体很多,但是常用的就那么几个。模板为CJK 常用的六种中文字体定义了简单易用的命令。它们是:

宋体:\verb|\songti|,CJK 等价命令\verb|\CJKfamily{song}|

黑体: \verb|\heiti|, CJK 等价命令 \verb|\CJKfamily{hei}|

仿宋: \verb|\fangsong|, CJK 等价命令 \verb|\CJKfamily{fs}|

楷书: \verb|\kaishu|, CJK 等价命令 \verb|\CJKfamily{kai}|

隶书: \verb|\lishu|, CJK 等价命令 \verb|\CJKfamily{li}|

幼圆: \verb|\youyuan|, CJK 等价命令 \verb|\CJKfamily{you}|

{\kaishu \TeX{} 系统中必须已经定义好这六种中文字体,并且使用和 \CTeX{} 套装中
一致的字体名称。(参见上面 CJK 等价命令的参数)

可用的字体命令还取决于使用的中文字库选项,参见 \ref{sec:fontset} 一节的介绍。

上面的字体在不同的字体选项下有不同的设置,不一定都有定义。}
\section{字号、字距、字宽和缩进}
中文字号的设置命令是\verb|\zihao{<字号>}|,例如\verb|\zihao{3}|。可以使用的参数有16 个,小号字体在前面加负号表示,从大到小依次为
\begin{center}
\begin{tabular}{cccccccc}
\toprule
初号 & 小初 & 一号 & 小一 & 二号 & 小二 & 三号 & 小三 \\
0 & -0 & 1 & -1 & 2 & -2 & 3 & -3 \\
\hline
四号 & 小四 & 五号 & 小五 & 六号 & 小六 & 七号 & 八号 \\
4 & -4 & 5 & -5 & 6 & -6 & 7 & 8 \\
\bottomrule
\end{tabular}
\end{center}
\noindent 英文字体大小会始终保持和中文字体一致。

汉字字距的调整使用命令 \verb|\ziju{<字宽的倍数>}|。参数可以是任意的数字,
例如 \verb|\ziju{5}| 设置汉字字距为当前汉字字宽的 5 倍, \verb|\ziju{0.5}| 设置汉字
字距为当前汉字字宽的一半。这里的汉字字宽指的是实际汉字的宽度,
不包含字间间隔。该命令不影响英文字距。

使用\verb|\CTEXindent| 可以正常的缩进两个汉字字宽的距离,同时在汉字大小和字距改变的情况都可以自动修改
缩进距离。使用\verb|\CTEXnoindent|可以取消缩进。
\section{中文数字转换}
使用 \verb|ctex| 提供的 \verb|\CTEXnumber| 命令可以将阿拉伯数字转换为中文数字。该命令的格式为
\begin{quote}
\verb|\CTEXnumber{<result>}{<number>}|
\end{quote}
其中 \verb|<result>| 必须是一个 \TeX{} 宏的名字,不需要预先定义。
例如
\begin{quote}
\verb|\CTEXnumber{\test}{100002005}|
\end{quote}
则\verb|\test|中的内容就是“一亿零二千零五”(不包括引号)。类似有\verb|\CTEXdigits|命令,若将上面示例中的\verb|\CTEXnumber|替换成\verb|\CTEXdigits|, 则\verb|\test|中的内容是“一〇〇〇〇二〇〇五”(不包括引号)。

对于计数器来说,可以用以下一些命令:

\begin{compactitem}
\item \verb|\chinese{<counter>}|: 一, 二, 三, $\ldots$
\item \verb|\arabic{<counter>}|: 1, 2, 3, $\ldots$
\item \verb|\roman{<counter>}|: i, ii, iii, $\ldots$
\item \verb|\Roman{<counter>}|: I, II, III, $\ldots$
\item \verb|\alph{<counter>}|: a, b, c, $\ldots$
\item \verb|\Alph{<counter>}|: A, B, C, $\ldots$
\end{compactitem}
\section{数学环境}
模板还定义了一些和数学符号相关的命令,用以修改\TeX{}默认的数学符号。主要有以下一些
\begin{center}
\begin{tabular}{ccc}
\toprule
代码&描述&效果\\
\midrule
\verb|\me|& 自然对数的底,直立体& \me\\
\verb|\mi|& 虚数单位,直立体& \mi\\
\verb|\dif|& 微分算子,直立体& \dif\\
\verb|\VEC{}|& 粗斜体表示向量& \VEC{a}\\
\verb|\MATRIX{}|& 无衬线字体表示矩阵& \MATRIX{A}\\
\verb|\TENSOR{}|& 无衬线字体表示张量& \MATRIX{T}\\
\bottomrule
\end{tabular}
\end{center}
\chapter{杂七杂八的话}
\section{模板的目录结构}
模板主要包含以下一些文件
\begin{center}
\begin{tabular}
{rl}
\toprule
\texttt{SDUthesistemplate.tex}& 主文档;\\
\texttt{sduthesis.cls}& 适合山东大学学士学位论文要求的文档类;\\
\texttt{sduthesis-front-cover.def}& 封面部分的定义;\\
\texttt{sduthesis-statement.def}& 声明部分的定义;\\
\texttt{contents}目录& 用于存放\texttt{.tex}文件;\\
\texttt{figures}目录& 用于存放图档;\\
\texttt{fonts-file}目录& 用于存放可能需要的字体文件。\\
\bottomrule
\end{tabular}
\end{center}

对于用户来说,最简单的办法,是在写作的时候打开\texttt{contents}目录下的\texttt{titlepageinfo.tex}文件填写目录和扉页信息;打开\texttt{contents}目录下的\texttt{abstract.tex}文件书写摘要和关键词;打开\texttt{contents}目录下的\texttt{body.tex}文件书写正文;打开\texttt{contents}目录下的\texttt{appendix.tex}文件填写附录;以及如果有需要填写在导言区中的内容,可以打开\texttt{contents}目录下的\texttt{usersettings.tex}文件。

这份说明文档,就是按照这样的目录结构来书写的。
\section{引用}
\subsection{参考文献}
参考文献方面,模板没有提供任何支持,完全依赖\texttt{book}类。之所以这么做,是因为使用\LaTeX{}进行论文排版的人水平参差不一,在这里做过多的设置反而会让初学者感到困惑。事实上用\BibTeX{}对参考文献进行处理是容易的。
\subsection{对图表、章节、公式的引用}
和\LaTeXe{}的习惯完全一致,需要先用\verb|\label{}|命令做一个“标签”,然后用\verb|\ref{}|命令来引用。例如式\ref{Equ:emc2}是被成为质能方程的公式。
\begin{equation}
E = m c^2.
\label{Equ:emc2}
\end{equation}
\section{图形和表格}
这里澄清一个长久的误会。\LaTeX{}支持的不仅仅是\texttt{.eps}格式的图档,事实上常见的\texttt{.jpg}和\texttt{.png}甚至\texttt{.pdf}格式的图档\LaTeX{}都是支持的。因此为了在\LaTeX{}格式标记的文档中插入图片,无需将图片特意转换为\texttt{.eps}格式。

使用\texttt{graphicx}宏包提供的\verb|\includegraphics[options]{name}|命令可以插入图档。如图\ref{Fig:pdf}就是用命令\verb|\includegraphics[width = .8\textwidth]{SDULogo.pdf}|插入的\texttt{.pdf}格式图档。
\begin{figure}
[htbp]
\centering
\includegraphics[width = .4\textwidth]{SDULogo.pdf}
\caption{一个 \emph{pdf} 图片的例子}\label{Fig:pdf}
\end{figure}

建议使用标准的“三线表”代替Microsoft Office中常见的表格,充满方框的表格给人以舒服的感觉,很烦很讨厌。例如一个简单的三线表如表\ref{Tab:3lines}所示。
\begin{table}
[htbp]
\centering
\caption{三线表的示例}\label{Tab:3lines}
\begin{tabular}
{ccc}
\toprule
姓名& 学号& 爱好\\
\hline
Ch'en Meng& 00000001& \LaTeX{}\\
Chen Chiang& 00000002& 收集书籍\\
\bottomrule
\end{tabular}
\end{table}

\chapter{增强这个模板}
欢迎你将你的反馈发送到我的邮箱,\href{mailto:chenmeng0518+TeX@gmail.com}{chenmeng0518+TeX@gmail.com}. 请注意这并不是一个负责回答问题的邮箱。

若是你有兴趣和我一起增强这个模板的功能,也欢迎通过上述邮箱联系我。